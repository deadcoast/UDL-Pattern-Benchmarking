\documentclass[12pt,a4paper]{article}

\usepackage{amsmath}
\usepackage{amssymb}
\usepackage{amsthm}
\usepackage{graphicx}
\usepackage{hyperref}
\usepackage{algorithm}
\usepackage{algorithmic}

\theoremstyle{definition}
\newtheorem{definition}{Definition}[section]
\newtheorem{theorem}{Theorem}[section]
\newtheorem{lemma}[theorem]{Lemma}
\newtheorem{proposition}[theorem]{Proposition}
\newtheorem{corollary}[theorem]{Corollary}

\theoremstyle{remark}
\newtheorem{remark}{Remark}[section]
\newtheorem{example}{Example}[section]

\title{Mathematical Framework for UDL Rating}
\author{UDL Rating Framework Team}
\date{\today}

\begin{document}

\maketitle

\begin{abstract}
This document provides the complete mathematical foundation for the User Defined Language (UDL) Rating Framework. We define the UDL representation space, specify each quality metric as a formal mathematical function, prove key properties, and provide complexity analysis. Every rating computation in the system is traceable to the rigorous definitions presented here.
\end{abstract}

\tableofcontents
\newpage

\section{Introduction}

The UDL Rating Framework evaluates the quality of User Defined Languages through mathematically-grounded metrics. This document establishes the formal foundations that ensure every rating is objective, reproducible, and traceable.

\subsection{Motivation}

Traditional language quality assessment relies on subjective expert judgment. Our framework replaces this with formal mathematical functions that can be computed algorithmically and verified independently.

\subsection{Overview}

We define:
\begin{itemize}
    \item A formal representation space for UDLs
    \item Four quality metrics: Consistency, Completeness, Expressiveness, and Structural Coherence
    \item An aggregation function combining metrics into overall quality
    \item A confidence measure based on prediction entropy
\end{itemize}

\section{UDL Representation Space}

\begin{definition}[UDL Representation]
A User Defined Language is represented as a tuple $U = (T, G, S, R)$ where:
\begin{itemize}
    \item $T = \{t_1, t_2, \ldots, t_n\}$ is a finite set of tokens (terminal symbols)
    \item $G = (V, E)$ is a directed graph representing the grammar, with vertices $V$ (non-terminals) and edges $E$ (production rules)
    \item $S: T \rightarrow \text{Semantics}$ is a semantic mapping function
    \item $R$ is a set of constraints and rules
\end{itemize}
\end{definition}

\subsection{Token Structure}

Each token $t \in T$ is a tuple $(text, type, pos, line, col)$ where:
\begin{itemize}
    \item $text \in \Sigma^*$ is the token text (string over alphabet $\Sigma$)
    \item $type \in \{\text{KEYWORD}, \text{IDENTIFIER}, \text{OPERATOR}, \text{LITERAL}\}$
    \item $pos, line, col \in \mathbb{N}$ are position indicators
\end{itemize}

\subsection{Grammar Graph}

The grammar graph $G = (V, E)$ encodes production rules:
\begin{itemize}
    \item Each vertex $v \in V$ represents a non-terminal symbol
    \item Each edge $(v_i, v_j) \in E$ represents a production rule $v_i \rightarrow \alpha$ where $v_j$ appears in $\alpha$
\end{itemize}

\section{Quality Metrics}

\subsection{Consistency Metric}

\begin{definition}[Consistency]
The consistency metric measures internal coherence:
\[
\text{Consistency}(U) = 1 - \frac{|C| + |Y|}{|R| + 1}
\]
where:
\begin{itemize}
    \item $C$ is the set of contradictory rule pairs
    \item $Y$ is the set of cycles in the grammar graph
    \item $R$ is the set of production rules
\end{itemize}
\end{definition}

\begin{theorem}[Consistency Boundedness]
For any UDL $U$, $0 \leq \text{Consistency}(U) \leq 1$.
\end{theorem}

\begin{proof}
To be completed in task 4.
\end{proof}

\subsection{Completeness Metric}

\begin{definition}[Completeness]
The completeness metric measures construct coverage:
\[
\text{Completeness}(U) = \frac{|D|}{|R_{\text{req}}|}
\]
where:
\begin{itemize}
    \item $D$ is the set of defined constructs in $U$
    \item $R_{\text{req}}$ is the set of required constructs for the language class
\end{itemize}
\end{definition}

\begin{theorem}[Completeness Boundedness]
For any UDL $U$, $0 \leq \text{Completeness}(U) \leq 1$.
\end{theorem}

\begin{proof}
To be completed in task 5.
\end{proof}

\subsection{Expressiveness Metric}

\begin{definition}[Expressiveness]
The expressiveness metric measures language power:
\[
\text{Expressiveness}(U) = \frac{\text{Chomsky}(U) + \text{Complexity}(U)}{2}
\]
where:
\begin{itemize}
    \item $\text{Chomsky}(U) \in \{0, 0.33, 0.67, 1.0\}$ classifies the grammar in the Chomsky hierarchy
    \item $\text{Complexity}(U)$ approximates Kolmogorov complexity via compression
\end{itemize}
\end{definition}

\subsection{Structural Coherence Metric}

\begin{definition}[Structural Coherence]
The structural coherence metric measures organizational quality:
\[
\text{StructuralCoherence}(U) = 1 - \frac{H(G)}{H_{\max}}
\]
where:
\begin{itemize}
    \item $H(G) = -\sum_{d} p(d) \log_2 p(d)$ is the Shannon entropy of the degree distribution
    \item $H_{\max} = \log_2 |V|$ is the maximum possible entropy
\end{itemize}
\end{definition}

\section{Aggregation Function}

\begin{definition}[Overall Quality]
The overall quality score is computed as:
\[
Q(U) = \sum_{i=1}^{4} w_i \cdot m_i(U)
\]
where:
\begin{itemize}
    \item $w_i \geq 0$ are weights with $\sum_{i=1}^{4} w_i = 1$
    \item $m_i$ are the individual metrics
\end{itemize}
\end{definition}

\begin{theorem}[Aggregation Boundedness]
If $m_i(U) \in [0,1]$ for all $i$ and $\sum w_i = 1$ with $w_i \geq 0$, then $Q(U) \in [0,1]$.
\end{theorem}

\begin{proof}
To be completed in task 8.
\end{proof}

\section{Confidence Measure}

\begin{definition}[Confidence]
The confidence in a quality assessment is:
\[
C = 1 - \frac{H(p)}{H_{\max}}
\]
where:
\begin{itemize}
    \item $H(p) = -\sum_{i} p_i \log p_i$ is the Shannon entropy of the prediction distribution
    \item $H_{\max} = \log n$ is the maximum entropy for $n$ classes
\end{itemize}
\end{definition}

\section{Complexity Analysis}

\subsection{Time Complexity}

To be completed in implementation tasks.

\subsection{Space Complexity}

To be completed in implementation tasks.

\section{Worked Examples}

\subsection{Example 1: Simple Grammar}

To be completed in task 21.

\subsection{Example 2: Complex Grammar}

To be completed in task 21.

\section{Literature References}

\begin{thebibliography}{99}

\bibitem{chomsky1956}
Chomsky, N. (1956). Three models for the description of language. \emph{IRE Transactions on Information Theory}, 2(3), 113-124.

\bibitem{shannon1948}
Shannon, C. E. (1948). A mathematical theory of communication. \emph{Bell System Technical Journal}, 27(3), 379-423.

\bibitem{kolmogorov1965}
Kolmogorov, A. N. (1965). Three approaches to the quantitative definition of information. \emph{Problems of Information Transmission}, 1(1), 1-7.

\end{thebibliography}

\end{document}
